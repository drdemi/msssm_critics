\chapter{Introduction and Motivations}
\thispagestyle{fancy}


\ldots

\section{Cellular Automation}
A cellular automation (CA) primarily consists of
\begin{itemize}
\item a finite regular $d$-dimensional field/lattice,
\item a set of variables attached to each cell/site and
\item a set of rules that specify the time evolution of the states.
\end{itemize}
A secondary property of a CA is the fact that the evolution rules are local, i.e. the updating of a certain cell only requires information about the cell itself and its finite, bounded and well defined neighbourhood.

Further analysis of the above definitions show that a CA is deterministic, i.e. a given initial configuration will always evolve the same way. \emph{Probabilistic} cellular automata imply an external probability to drive the updating rule and therefore allow to introduce a sort of continuity, even though the automation is of discrete nature.

\section{Self-Organized Criticality}
The term self-organized criticality (SOC) basically consists of two properties:
\begin{itemize}
\item \emph{self-organization} means that a non-equilibrium system is able to develop structures on its own, without external control or manipulation.
\item \emph{criticality} implies that a local disturbance not only influences the local neighbourhood, but the whole system. In other words, all the members of a system influence each other. This term originally comes from thermodynamics and describes a state at the phase transition, where a substance (e.g. water) resides between different phases.
\end{itemize}

\section{CA and SOC}
Generally it is difficult to determine whether a certain self-organized system exhibits self-organized criticality. For certain cases 
