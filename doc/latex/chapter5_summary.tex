\chapter{Summary and Outlook}
\thispagestyle{fancy}

\section{Conclusion}
SOC is still a vague phenomenology, involving many attempts to use it for explaining power-like distributions in real systems using computational models.

From the studies of different cases of the abelian sandpile, it can be concluded that the continuous case, with a realistic field and friction in propagating the grains, the system bears a nice power-law like distribution for avalanche sizes, with a little worse fits for avalanche lifetimes. This case, independent of the boundary conditions, is also the more realistic one, therefore one might identify it with SOC. 

In summary, \textbf{slowly driven} systems are associated with \textbf{dissipation} and a \textbf{local threshold} that leads to avalanche phenomena and SOC. The computational models provided in this paper bear these characteristics. Distinct specific laws might change only the critical exponent of the power-law distribution. 

The cellullar automation models of sandpile is a convenient way of modeling real systems with the properties discussed.

A deeper understanding of SOC, if it exists, definitely needs a mathematical framework (???).
