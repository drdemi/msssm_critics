%!TEX root = bericht.tex
%!TEX encoding = latin1

%---------------------------------------------------------------------------
% Abstract



\phantomsection
\addcontentsline{toc}{chapter}{Abstract}
\chapter*{Abstract}

This paper describes the principles of self-organized criticality and their validity in cellular automation models, in particular the sandpile model. The model, its diversity, its implementation in MATLAB/Octave with different parameters and its analysis are presented in detail. Different aspects to the nature of critical systems, such as fractal structure and power-law distributions, are discussed including the effect of different system parameters, such as field size, its dimension, its boundary, presence of friction or the decoupling of the driving and the avalanche time scales.


\startnewchapter

% blank page
%\newpage
%\thispagestyle{empty}
%\mbox{}

%---------------------------------------------------------------------------
% Dankssagung

\phantomsection
\addcontentsline{toc}{chapter}{Acknowledgements}
\chapter*{Acknowledgements}

The project group would like to thank the \emph{Chair of Sociology, in particular of Modeling and Simulation}, for the chance to work on an interesting project and the possibility to apply simulation skills on extraordinary topics. Special thanks go to the assistants, namely Karsten Donnay and Stefano Balietti, for their all-time support during the project. Furthermore, the group would like to thank Pegah Kassraian Fard, who unfortunately had to quit the project, for her support in the early state of the project. Additional thanks go to the other groups of the class and of previous semesters.

\startnewchapter

%---------------------------------------------------------------------------
% Symbols

%\phantomsection
%\addcontentsline{toc}{chapter}{Nomenklatur}
%\chapter*{Nomenklatur}
%\label{chp:nomenklatur}

%\section*{Abk�rzungen}
%\begin{tabbing}
%\hspace*{1.6cm}\=\kill

%STC\> standard testing conditions\\[0.5ex]
%NX\> CAD, CAE und CAM Softwarepaket von Siemens

%\end{tabbing}


% declaration of originality
%\includepdf{../confirmation_en.pdf}


\startnewchapter

%---------------------------------------------------------------------------
% Table of contents

\setcounter{tocdepth}{2}
% create a bookmark that will appear in the booksmarks displayed by
% Adobe Reader, but not in the Table of Contents of the document
\label{TOC}
\pdfbookmark[0]{Table of contents}{TOC}
\tableofcontents


\startnewchapter

%---------------------------------------------------------------------------
